\documentclass[Orbiter User Manual.tex]{subfiles}
\begin{document}

\section{Installation}
\label{sec:installation}
This section lists the computer requirements for running Orbiter, and contains download and installation instructions.

\subsection{Computer requirements}
Orbiter should run on most Windows PCs running a recent Windows version.\\
The following table lists the hardware requirements needed by Orbiter.

%\begin{table}[H]
	%\centering
	\begin{longtable}{ |p{0.065\textwidth}|p{0.3\textwidth}|p{0.55\textwidth}| }
	\hline\rule{0pt}{2ex}
	& \textbf{Minimum requirements} & \textbf{Recommended requirements}\\
	\hline\rule{0pt}{2ex}
	RAM & 500MB & 2GB\\
	\hline\rule{0pt}{2ex}
	CPU & Dual Core &\\
	\hline\rule{0pt}{2ex}
	GPU & 50 GFlops & 100 GFlops\\
	\hline\rule{0pt}{2ex}
	Disk & 5GB of free space & 10GB of free space (80GB if you want hi-res textures)\\
	\hline
	\end{longtable}
%\end{table}

\noindent
Orbiter supports an optional joystick.\\
The TrackIR head tracker is supported via a plug-in module included in the basic installation.

\subsection{Download}
The latest Orbiter release can be downloaded from \url{https://github.com/orbitersim/orbiter/releases}.\\
If desired, high-resolution texture packs are available for Mercury, Earth, Moon, Mars, Titan and some minor-bodies, which can be downloaded from \url{http://orbit.medphys.ucl.ac.uk/mirrors/orbiter_radio/tex_mirror.html}. The downloads are provided as torrents for fast and reliable transfer. If you can't handle torrent downloads, alternative http download links are also provided.

\subsection{Installation}
\begin{itemize}
\item Create a new folder for the Orbiter installation, e.g. C:\textbackslash Orbiter or \%HOMEPATH\%\textbackslash Orbiter. Note that creating an Orbiter folder in Program files or Program files (x86) is not recommended, because Windows puts some restrictions on these locations.
\item If a previous version of Orbiter is already installed on your computer, you should not install the new version into the same folder, because this could lead to file conflicts. You may want to keep your old installation until you have made sure that the latest version works without problems. Multiple Orbiter installations can exist on the same computer.
\item Unzip the Orbiter ZIP installation package into the new folder, using either the default Windows unzip function, or an external tool like 7-zip or WinZip. Important: Take care to preserve the directory structure of the package (for example, in WinZip this requires to activate the "Use Folder Names" option).
\item After unzipping the package, make sure your Orbiter folder contains the executables (orbiter.exe and orbiter\_ng.exe) and, among other files, the Config, Meshes, Scenarios and Textures subfolders.

%TODO handle redist dependencies "Microsoft Visual C++ 2015 - 2022 Redistributable"

\item The Orbiter simulator can be launched either with a built-in graphics engine (orbiter.exe, with the red "Delta-glider" icon), or with an external graphics interface (orbiter\_ng.exe, with the blue "Delta-glider" icon). The second option allows to connect to external graphics engines with enhanced features and performance. This requires downloading and installing 3rd party Orbiter graphics engines, or using the included D3D9Client plugin. Once running, Orbiter will show you the "Launchpad" dialog, where you can select video options and simulation parameters.
\item You are now ready to start, select a scenario from the Launchpad dialog and click the \textit{Launch Orbiter} button!
\end{itemize}

\noindent
\alertbox{If Orbiter does not show any scenarios in the Scenario tab, or if planets appear plain white without any textures when running the simulation, the most likely reason is that the installation packages were not properly unpacked. Make sure your Orbiter folder contains the subfolders as described above. If necessary, you may have to repeat the installation process.}

\subsection{Uninstall}
Simply remove the Orbiter folders with all contents and subdirectories. This will completely remove Orbiter from your hard drive.

\end{document}
