\documentclass[Orbiter Developer Manual.tex]{subfiles} 
\begin{document}

\section{The vessel class configuration file}
Configuration files are ASCII text files which can be edited with any text editor capable of writing plain text files (e.g notepad). All vessel configuration files are by default located in ORBITER’s Config subdirectory (unless the \textit{ConfigDir} entry in Orbiter.cfg points to a different directory).\\
\\
Below is a description of the default vessel configuration options recognised by Orbiter. Note that not all options need to be present in a configuration file. In particular vessels defined via costomised modules may specify various parameters directly in the module. Furthermore, vessel modules may read additional custom parameters not listed here from the configuration file.

%\begin{table}[H]
	%\centering
	\begin{longtable}{ |p{0.25\textwidth}|p{0.07\textwidth}|p{0.6\textwidth}| }
	\hline\rule{0pt}{2ex}
	\textbf{Item} & \textbf{Type} & \textbf{Description}\\
	\hline\rule{0pt}{2ex}
	BaseClass & S & Optional; parent class. Missing entries are taken from this class. Allows the construction of class hierarchies. (Make sure not to introduce circular dependencies!)\\
	\hline\rule{0pt}{2ex}
	Module & S & Optional; name of plugin module for vessel customisation. The module must be located in the Modules folder.\\
	\hline\rule{0pt}{2ex}
	Help & S,S & Optional; name of help file to be used for vessel class specific help when the user presses the “Vessel” button on the Help dialog. The help file must be a compiled html file (.CHM) and be located in directory Html/Vessels. The entry contains the file name without path and extension, and (separated by comma) the name of the first page of the file to be displayed (without extension). Default: no vessel class specific help.\\
	\hline\rule{0pt}{2ex}
	EditorCreate & B & If \textit{false}, the vessel type does not appear in the list on the vessel creation page of the scenario editor. (default: \textit{true})\\
	\hline\rule{0pt}{2ex}
	ImageBmp & S & File name of a bitmap file (BMP) displaying the vessel. The name should include the path (relative to orbiter main directory) and extension (.bmp). This image is shown on the vessel creation page of the Scenario Editor. For best results, it should be size 164x240 pixels.\\
	\hline\rule{0pt}{2ex}
	MeshName & S & Name of the mesh used for visualisation\\
	\hline\rule{0pt}{2ex}
	EnableFocus & B & \textit{true} if vessel can receive input focus (default: \textit{true})\\
	\hline\rule{0pt}{2ex}
	EnableXPDR & B & \textit{true} if vessel carries a transponder (default: \textit{false})\\
	\hline\rule{0pt}{2ex}
	XPDR & I & transponder channel (in units of 0.05 kHz from 108.0 kHz). Only used if EnableXPDR=true. This default channel may be overridden by a vessel’s scenario script.\\
	\hline\rule{0pt}{2ex}
	Mass & F & Vessel mass (empty) [kg]\\
	\hline\rule{0pt}{2ex}
	Size & F & (Mean) vessel radius [m]\\
	\hline\rule{0pt}{2ex}
	MaxMainThrust & F & Main thruster rating [N]\\
	\hline\rule{0pt}{2ex}
	MaxRetroTrust & F & Retro thruster rating [N]\\
	\hline\rule{0pt}{2ex}
	MaxHoverThrust & F & Hover thruster rating [N]\\
	\hline\rule{0pt}{2ex}
	MaxAttitudeThrust & F & Thrust rating for reaction contol engines [N]\\
	\hline\rule{0pt}{2ex}
	TouchdownPoints & V V V & 3 surface contact points in local vessel coordinates. For aircraft-like configurations these are: nose wheel, left main wheel, right main wheel. (the order is important to define the “up” direction). Other spacecraft types may interpret the points differently.\\
	\hline\rule{0pt}{2ex}
	CameraOffset & V & Camera position inside the vessel for cockpit view\\
	\hline\rule{0pt}{2ex}
	CW & F F F F & Airflow resistance coefficients: forward, backward, transversal, vertical. Only used by legacy flight model (if no airfoils are defined in the module).\\
	\hline\rule{0pt}{2ex}
	WingAspect & F & The wing aspect ratio (wingspan2 / wing area). Used for atmospheric drag calculation in the legacy flight model.\\
	\hline\rule{0pt}{2ex}
	WingEffectiveness & F & A wing form factor: ~3.1 for elliptic wings, ~2.8 for tapered wings, ~2.5 for rectangular wings. Only used by legacy flight model.\\
	\hline\rule{0pt}{2ex}
	CrossSections & V & Cross sections in axis directions (z=longitudinal) [m$^{2}$]\\
	\hline\rule{0pt}{2ex}
	RotResistance & V & Resistance against rotation around axes in atmosphere, where angular deceleration due to atmospheric friction is $a_{x,y,z}^{(\omega)} = -v_{x,y,z}^{(\omega)} \, \rho \, r_{x,y,z}$ with angular velocity $v^{(\omega)}$ and atmospheric density $\rho$.\\
	\hline\rule{0pt}{2ex}
	Inertia & V & Principal moments of inertia, mass-normalised (see below) [m$^{2}$]\\
	\hline\rule{0pt}{2ex}
	GravityGradientDamping & F & Damping coefficient for gravity gradient torque. Determines relaxation time for tidal locking. Default: 0 (undamped).\\
	\hline\rule{0pt}{2ex}
	PropellantResource\textit{i} & F [F] & Specs for propellant resource \textit{i} ($i \geq 1$). First value: max. fuel capacity [kg]. Second value: fuel efficiency factor (>0, default: 1)\\
	\hline\rule{0pt}{2ex}
	MaxFuel & F & Max. fuel mass [kg]. Obsolete; only used if no propellant resources are defined\\
	\hline\rule{0pt}{2ex}
	Isp & F & Default value for fuel-specific impulse [m/s]: Amount of thrust [N] obtained by burning 1kg of fuel per second. Vessel modules can override this value for individual engines.\\
	\hline\rule{0pt}{2ex}
	MEngineRef\textit{i} & V & Reference position for main thruster i (i=1...)\\
	\hline\rule{0pt}{2ex}
	REngineRef\textit{i} & V & Reference position for retro thruster i (i=1...)\\
	\hline\rule{0pt}{2ex}
	HEngineRef\textit{i} & V & Reference position for hover thruster i (i=1...)\\
	\hline\rule{0pt}{2ex}
	AttRefd\textit{ij} & V & Reference position for attitude thruster (for rotation around axis d (d=X,Y,Z), rotation direction i (i=1,2) and thruster index j (j=1,2)) for a total of 12 attitude thrusters\\
	\hline\rule{0pt}{2ex}
	LongAttRef\textit{ij} & V & Reference position for attitude thrusters (for linear forward/backward translation), direction i (i=1,2) and thruster index j (j=1,2)) for a total of 4 attitude thrusters\\
	\hline\rule{0pt}{2ex}
	DockRef & V & Docking reference point for first docking port (obsolete)\\
	\hline\rule{0pt}{2ex}
	DockDir & V & Docking approach direction for first docking port (obsolete)\\
	\hline\rule{0pt}{2ex}
	DockRot & V & Longitudinal alignment direction (normal to DockDir) for first docking port (obsolete)\\
	\hline\rule{0pt}{2ex}
	\textit{<Docklist>} & List & List of positions and approach directions for docking ports (see below).\\
	\hline\rule{0pt}{2ex}
	\textit{<Attachment list>} & List & List of positions and approach directions for attachment points (see below).\\
	\hline
	\end{longtable}
%\end{table}
\noindent
(S=String, B=Bool, F=Float, V=Vector)\\
\\
\underline{Notes:}

\begin{itemize}
\item A vessel class can be derived from a different vessel class, by defining the BaseClass entry. All properties not defined in the new class configuration file are taken from the base class.
\item The mesh name should not contain the file extension (.msh)  and should not contain a directory path.
\item The MaxFuel entry has been replaced by PropellantResource, which allows the definition of multiple propellant resources (fuel tanks).
\item The DockRef, DockDir, DockRot entries have been replaced with the more versatile Docklist (see below), which allows the configuration of multiple docking ports and IDS frequencies.
%TODO
\item The attachment list is similar to the docklist: it allows to specify points at which vessels can be connected to each other. Unlike docking ports, attachment points define parent-child hierarchies, and each attachment point is either a parent or a child port. For more details see the Vessel attachment management section in the API Reference Manual.
%TODO
\item Inertia tensor J: Relates angular momentum and angular velocity:
%TODO
\end{itemize}

\noindent
\alertbox{The SDK contains a simple tool to calculate the inertia tensor for a given mesh: Orbitersdk\textbackslash utils\textbackslash shipedit.exe. The tool requires “well behaved” meshes (composed of closed compact surfaces) and assumes a homogeneous density distribution inside the mesh. The latter is not very realistic, so the results must be interpreted carefully. They should still serve as a good starting point for experimentation.}


\section{Configuration files for individual vessels}
A vessel only requires an individual definition file if it is not an instance of a vessel class. In this case the format for the vessel’s .cfg file is identical to the vessel class .cfg files described above.


\end{document}
